% !TEX root = ../thesis.tex

\chapter{Úvod}\label{ch:introduction}

V~súčasnom dynamickom prostredí softvérového inžinierstva sa projektové riadenie stalo kritickým faktorom úspechu pre organizácie všetkých veľkostí. S~rastúcou komplexnosťou softvérových systémov a~skracovaním vývojových cyklov vznikla potreba sofistikovaných nástrojov, ktoré dokážu efektívne koordinovať prácu vývojových tímov, sledovať progres projektov a~zabezpečiť transparentnú komunikáciu medzi všetkými zainteresovanými stranami~\cite{sutherland2014}.

Agilné metodiky, predovšetkým Scrum a~Kanban, sa v~posledných dvoch dekádach etablovali ako dominantné prístupy k~vývoju softvéru. Podľa prieskumu State of Agile Report využíva agilné metodiky viac ako 94\% organizácií pôsobiacich v~oblasti softvérového vývoja~\cite{stateofagile2023}. Tento masívny prechod k~agilným prístupom vyvolal potrebu špecializovaných nástrojov, ktoré dokážu podporiť iteratívny a~inkrementálny charakter agilného vývoja.

Medzi najrozšírenejšie nástroje pre riadenie agilných projektov patrí Jira od spoločnosti Atlassian. Od svojho uvedenia na trh v~roku 2002 sa Jira vyvinula z~jednoduchého systému na sledovanie chýb na komplexnú platformu pre projektové riadenie, ktorú využíva viac ako 180~000 organizácií po celom svete~\cite{atlassian2024}. Jej flexibilita, rozsiahle možnosti prispôsobenia a~robustný ekosystém integrácií z~nej urobili štandard v~odvetví informačných technológií.

\section{Motivácia a ciele práce}

Hlavným cieľom tohto referátu je poskytnúť komplexnú analýzu nástroja Jira z~pohľadu jeho využitia v~riadení agilných projektov. Práca sa zameriava na:

\begin{itemize}
    \item detailnú charakteristiku nástroja a~jeho technickej architektúry,
    \item analýzu kľúčových funkcionalít a~ich praktického využitia v~projektovom cykle,
    \item kritické zhodnotenie výhod a~limitácií nástroja,
    \item porovnanie s~konkurenčnými riešeniami na trhu.
\end{itemize}

Referát je primárne určený pre projektových manažérov, Scrum Masterov, Product Ownerov a~vývojárov, ktorí zvažujú implementáciu alebo optimalizáciu využívania Jiry vo svojich organizáciách. Sekundárnou cieľovou skupinou sú študenti informatiky a~softvérového inžinierstva, ktorí sa pripravujú na vstup do profesionálnej praxe.

\section{Štruktúra práce}

Referát je štruktúrovaný do siedmich hlavných kapitol. Po úvodnej kapitole nasleduje druhá kapitola venovaná všeobecnej charakteristike nástroja Jira, jeho histórii a~cieľovej skupine používateľov. Tretia kapitola sa zaoberá technickými aspektmi a~architektúrou systému. Štvrtá kapitola predstavuje jadro práce a~detailne analyzuje využitie Jiry v~jednotlivých fázach projektového cyklu. Piata kapitola sumarizuje hlavné prínosy nástroja, zatiaľ čo šiesta kapitola poskytuje kritické zhodnotenie výhod a~nevýhod. Siedma kapitola obsahuje porovnávaciu analýzu s~konkurenčnými nástrojmi. Prácu uzatvára záverečná kapitola so zhrnutím kľúčových zistení.
