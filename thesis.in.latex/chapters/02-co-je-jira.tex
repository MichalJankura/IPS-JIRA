% !TEX root = ../thesis.tex

\chapter{Čo je Jira?}\label{ch:co-je-jira}

Jira je komplexný softvérový nástroj pre riadenie projektov, sledovanie úloh a koordináciu tímovej spolupráce, vyvinutý austrálskou spoločnosťou Atlassian. Názov produktu je odvodený od japonského slova \textit{Gojira}, čo je pôvodný názov pre filmovú príšeru Godzillu. Toto pomenovanie vzniklo ako interná prezývka pre konkurenčný produkt Bugzilla a následne bolo adoptované pre nový projekt \cite{atlassian_history2023}.

\section{História a vývoj nástroja}

Spoločnosť Atlassian založili Mike Cannon-Brookes a~Scott Farquhar v~roku 2002 v~Sydney v~Austrálii \cite{atlassian_company2024}. Ešte v~tom istom roku bola vydaná prvá verzia Jiry~(1.0), pôvodne navrhnutá ako systém na sledovanie chýb (bug tracker).

Evolúcia Jiry prešla niekoľkými kľúčovými fázami. V~počiatočnom období (2002\,--\,2005) sa produkt postupne rozšíril z~jednoduchého bug trackera o~workflow engine a~pokročilé možnosti konfigurácie. Prelomovým sa stal rok 2009, keď Atlassian akvizíciou pluginu GreenHopper získal podporu pre metodiky Scrum a~Kanban; súbežne boli sprístupnené prvé hostované verzie Jiry a~Confluence. V~roku 2011 nasledovalo spustenie platformy Atlassian OnDemand ako plnohodnotnej SaaS služby a~v~roku 2013 uvedenie produktu Jira Service Desk pre oblasť IT service managementu.

Rok 2015 priniesol rozdelenie na tri špecializované produktové línie~-- Jira Software pre vývojové tímy, Jira Core pre netechnické tímy a~Jira Service Desk pre zákaznícku podporu. Portfólio spoločnosti sa ďalej rozrástlo v~roku 2017 akvizíciou vizuálneho Kanban nástroja Trello a~v~roku 2021 premenovaním Jira Core na Jira Work Management so zameraním na biznis tímy.

Strategický obrat k~modelu cloud-first začal v~roku 2020 oznámením ukončenia podpory pre Server verziu; v~tom istom roku bola Jira Service Desk premenovaná na Jira Service Management. Prechod bol zavŕšený vo februári 2024 definitívnym ukončením podpory Server licencií a~zlúčením Jira Software a~Jira Work Management do jednotného produktu s~názvom Jira \cite{atlassian_server_end2024}.

Vývoj Jiry ilustruje Obr.~\ref{fig:timeline}, ktorý zachytáva kľúčové milníky od prvého vydania v~roku 2002 až po zlúčenie produktových línií v~roku 2024.

\begin{figure}[ht]
\centering
% TODO: vložiť obrázok -- časová os (timeline) zobrazujúca kľúčové milníky vývoja Jiry:
% 2002 Jira 1.0, 2009 GreenHopper/Scrum+Kanban, 2011 OnDemand SaaS, 2013 Service Desk,
% 2015 rozdelenie na Software/Core/Service Desk, 2017 akvizícia Trello, 2020 cloud-first,
% 2024 zlúčenie do Jira. Zdroj: vlastná schéma.
\fbox{\parbox{0.8\textwidth}{\centering\vspace{2cm}[TODO: vložiť obrázok]\vspace{2cm}}}
\caption{Časová os vývoja nástroja Jira od roku 2002 po súčasnosť}
\label{fig:timeline}
\end{figure}

\section{Filozofia a prístup nástroja}

Jira je postavená na filozofii flexibility a prispôsobiteľnosti. Na rozdiel od rigidných nástrojov, ktoré vnucujú používateľom konkrétnu metodiku, Jira poskytuje stavebné bloky, ktoré si organizácie môžu konfigurovať podľa vlastných procesov. Tento prístup vychádza z poznania, že neexistuje univerzálne riešenie projektového riadenia -- každý tím má jedinečné potreby a pracovné postupy \cite{cohn2010}.

Návrh Jiry sa opiera o~štyri základné princípy. Prvým je \textit{transparentnosť}~-- všetky informácie o~projektoch sú prístupné relevantným členom tímu, čo podporuje zdieľané porozumenie a~znižuje informačnú asymetriu. S~tým úzko súvisí \textit{sledovateľnosť}: každá zmena je zaznamenaná a~auditovateľná, čo uľahčuje spätnú analýzu rozhodnutí aj plnenie regulačných požiadaviek. Tretím princípom je \textit{flexibilita}~-- systém sa prispôsobuje procesom organizácie, nie naopak, čo umožňuje adopciu bez vynútenia konkrétnej metodiky. Napokon, \textit{integrovateľnosť} zabezpečuje, že otvorená architektúra a~rozsiahle API umožňujú prepojenie Jiry s~externými nástrojmi a~službami.

\section{Cieľová skupina používateľov}

Jira slúži širokému spektru používateľov v rámci softvérového vývojového procesu. Každá používateľská rola interaguje s nástrojom odlišným spôsobom a využíva rozdielne funkcionality.

\subsection{Vývojári softvéru}

Pre vývojárov predstavuje Jira primárny zdroj informácií o~pridelených úlohách. V~každodennej praxi ju využívajú na sledovanie pridelených issue a~ich priorít, dokumentáciu technických detailov a~riešení, ako aj na prepojenie commitov a~pull requestov priamo s~konkrétnymi úlohami. Dôležitú úlohu zohráva aj kontextová komunikácia prostredníctvom komentárov pri jednotlivých úlohách a~logovanie odpracovaného času, ktoré umožňuje presnejšie plánovanie kapacity.

\subsection{Projektoví manažéri a Scrum Masteri}

Projektoví manažéri a~Scrum Masteri využívajú Jiru predovšetkým na koordináciu a~plánovanie. Kľúčovými aktivitami sú vytváranie a~prioritizácia backlogu, plánovanie šprintov a~verzií a~sledovanie kapacity tímu. Na základe automaticky generovaných reportov a~metrík dokážu identifikovať blokujúce faktory a~operatívne na ne reagovať, čím udržiavajú plynulý priebeh projektu.

\subsection{Product Owneri}

Pre Product Ownerov je Jira nástrojom na definovanie a~prioritizáciu požiadaviek, vytváranie user stories a~technických špecifikácií. Progres jednotlivých funkcionalít je možné priebežne sledovať a~komunikovať so stakeholdermi prostredníctvom zdieľaných dashboardov, ktoré poskytujú prehľad bez nutnosti vstupovať do detailov implementácie.

\subsection{Vedenie a stakeholderi}

Na úrovni manažmentu slúži Jira predovšetkým ako strategický nástroj. Umožňuje dlhodobé plánovanie pomocou roadmáp, sledovanie kľúčových metrík a~KPI a~získavanie agregovaného prehľadu o~stave projektov bez potreby detailného ponorenia do operačných detailov. Rozhodovanie sa tak môže opierať o~aktuálne dáta z~reportov namiesto subjektívnych hodnotení.

Vzájomnú interakciu jednotlivých rolí s~nástrojom Jira zachytáva Obr.~\ref{fig:roles}. Z~diagramu je zrejmé, že každá rola pristupuje k~iným častiam systému a~využíva odlišné funkcionality, no všetky sa stretávajú v~spoločnom dátovom modeli projektov a~úloh.

\begin{figure}[ht]
\centering
% TODO: vložiť obrázok -- diagram zobrazujúci interakciu rolí (vývojári, Scrum Master,
% Product Owner, manažment) s Jirou: ktoré funkcionality využíva každá rola,
% kde sa prekrývajú (napr. dashboardy, komentáre). Zdroj: vlastná schéma.
\fbox{\parbox{0.8\textwidth}{\centering\vspace{2cm}[TODO: vložiť obrázok]\vspace{2cm}}}
\caption{Schéma interakcie rôznych používateľských rolí s~nástrojom Jira}
\label{fig:roles}
\end{figure}

\section{Produktové línie Jira}

V súčasnosti Atlassian ponúka niekoľko špecializovaných verzií Jiry, každú optimalizovanú pre konkrétne použitie:

\textbf{Jira} (predtým Jira Software) je vlajková loď produktovej línie, určená pre softvérové vývojové tímy. Obsahuje natívnu podporu pre Scrum a Kanban, integráciu s vývojárskymi nástrojmi a pokročilé reportingové možnosti. V roku 2024 boli Jira Software a Jira Work Management zlúčené do jedného produktu s názvom Jira \cite{atlassian_jira_software2024}.

\textbf{Jira Service Management} (predtým Jira Service Desk) je určená pre IT service management a zákaznícku podporu. Implementuje ITIL framework a poskytuje funkcionalitu pre incident management, change management a self-service portály.

Historicky existovala aj \textbf{Jira Work Management} (predtým Jira Core), zjednodušená verzia pre netechnické tímy ako marketing, HR alebo právne oddelenia. Od roku 2024 je táto funkcionalita integrovaná priamo do hlavného produktu Jira.

Pre účely tohto referátu sa budeme primárne zameriavať na Jiru, keďže tento produkt je najrelevantnejší pre riadenie agilných softvérových projektov.
