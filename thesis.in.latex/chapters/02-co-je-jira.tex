% !TEX root = ../thesis.tex

\chapter{Čo je Jira?}\label{ch:co-je-jira}

Jira je komplexný softvérový nástroj pre riadenie projektov, sledovanie úloh a~koordináciu tímovej spolupráce, vyvinutý austrálskou spoločnosťou Atlassian. Názov produktu je odvodený od japonského slova \textit{Gojira}, čo je pôvodný názov pre filmovú príšeru Godzillu. Toto pomenovanie vzniklo ako interná prezývka pre konkurenčný produkt Bugzilla a~následne bolo adoptované pre nový projekt~\cite{atlassian_history2023}.

\section{História a vývoj nástroja}

Spoločnosť Atlassian bola založená v~roku 2002 v~Sydney, Austrália, spoluzakladateľmi Mikeom Cannonom-Brookesom a~Scottom Farquharom. V~tom istom roku bola vydaná prvá verzia Jiry, pôvodne navrhnutá ako nástroj na sledovanie chýb (bug tracker) pre interné potreby spoločnosti~\cite{atlassian_company2024}.

Evolúcia Jiry prešla niekoľkými kľúčovými míľnikmi:

\begin{itemize}
    \item \textbf{2002} -- Vydanie prvej verzie Jira 1.0 ako systému na sledovanie chýb
    \item \textbf{2005} -- Rozšírenie funkcionality o~workflow engine a~customizáciu
    \item \textbf{2009} -- Predstavenie Jira Agile (pôvodne GreenHopper) pre podporu Scrum a~Kanban
    \item \textbf{2012} -- Spustenie Jira Cloud ako SaaS riešenia
    \item \textbf{2015} -- Integrácia agilných funkcionalít priamo do core produktu
    \item \textbf{2017} -- Predstavenie Jira Software, Jira Service Management a~Jira Work Management ako špecializovaných produktových línií
    \item \textbf{2020} -- Oznámenie ukončenia podpory pre Server verziu a~prechod na Cloud a~Data Center
    \item \textbf{2024} -- Definitívne ukončenie predaja Server licencií~\cite{atlassian_server_end2024}
\end{itemize}

[Obr. 1: Časová os vývoja nástroja Jira od roku 2002 po súčasnosť]

\section{Filozofia a prístup nástroja}

Jira je postavená na filozofii flexibility a~prispôsobiteľnosti. Na rozdiel od rigidných nástrojov, ktoré vnucujú používateľom konkrétnu metodiku, Jira poskytuje stavebné bloky, ktoré si organizácie môžu konfigurovať podľa vlastných procesov. Tento prístup vychádza z~poznania, že neexistuje univerzálne riešenie projektového riadenia -- každý tím má jedinečné potreby a~pracovné postupy~\cite{cohn2010}.

Základné princípy návrhu Jiry zahŕňajú:

\begin{enumerate}
    \item \textbf{Transparentnosť} -- Všetky informácie o~projektoch sú dostupné relevantným členom tímu
    \item \textbf{Sledovateľnosť} -- Každá zmena je zaznamenaná a~auditovateľná
    \item \textbf{Flexibilita} -- Systém sa prispôsobuje procesom, nie naopak
    \item \textbf{Integrovateľnosť} -- Otvorená architektúra umožňuje prepojenie s~externými nástrojmi
\end{enumerate}

\section{Cieľová skupina používateľov}

Jira slúži širokému spektru používateľov v~rámci softvérového vývojového procesu. Každá používateľská rola interaguje s~nástrojom odlišným spôsobom a~využíva rozdielne funkcionality.

\subsection{Vývojári softvéru}

Pre vývojárov predstavuje Jira primárny zdroj informácií o~pridelených úlohách. Vývojári využívajú nástroj na:
\begin{itemize}
    \item sledovanie pridelených issue a~ich priorít,
    \item dokumentáciu technických detailov a~riešení,
    \item prepojenie commitov a~pull requestov s~úlohami,
    \item komunikáciu s~ostatnými členmi tímu prostredníctvom komentárov,
    \item logovanie odpracovaného času.
\end{itemize}

\subsection{Projektoví manažéri a Scrum Masteri}

Táto skupina využíva Jiru predovšetkým na koordináciu a~plánovanie:
\begin{itemize}
    \item vytváranie a~prioritizáciu backlogu,
    \item plánovanie šprintov a~verzií,
    \item sledovanie kapacity tímu,
    \item generovanie reportov a~metrík,
    \item identifikáciu a~riešenie blokujúcich faktorov.
\end{itemize}

\subsection{Product Owneri}

Product Owneri využívajú Jiru na:
\begin{itemize}
    \item definovanie a~prioritizáciu požiadaviek,
    \item vytváranie user stories a~špecifikácií,
    \item sledovanie progresu jednotlivých funkcionalít,
    \item komunikáciu so stakeholdermi prostredníctvom zdieľaných dashboardov.
\end{itemize}

\subsection{Vedenie a stakeholderi}

Na úrovni manažmentu slúži Jira ako nástroj pre:
\begin{itemize}
    \item strategické plánovanie pomocou roadmáp,
    \item sledovanie kľúčových metrík a~KPI,
    \item získavanie prehľadu o~stave projektov bez potreby detailného ponorenia,
    \item rozhodovanie na základe dát z~reportov.
\end{itemize}

[Obr. 2: Schéma interakcie rôznych používateľských rolí s~nástrojom Jira]

\section{Produktové línie Jira}

V~súčasnosti Atlassian ponúka niekoľko špecializovaných verzií Jiry, každú optimalizovanú pre konkrétne použitie:

\textbf{Jira Software} je vlajková loď produktovej línie, určená pre softvérové vývojové tímy. Obsahuje natívnu podporu pre Scrum a~Kanban, integráciu s~vývojárskymi nástrojmi a~pokročilé reportingové možnosti~\cite{atlassian_jira_software2024}.

\textbf{Jira Service Management} (predtým Jira Service Desk) je určená pre IT service management a~zákaznícku podporu. Implementuje ITIL framework a~poskytuje funkcionalitu pre incident management, change management a~self-service portály.

\textbf{Jira Work Management} (predtým Jira Core) je zjednodušená verzia pre netechnické tímy ako marketing, HR alebo právne oddelenia. Poskytuje intuitívne rozhranie bez komplexnosti typickej pre softvérový vývoj.

Pre účely tohto referátu sa budeme primárne zameriavať na Jira Software, keďže táto verzia je najrelevantnejšia pre riadenie agilných softvérových projektov.
