% !TEX root = ../thesis.tex

\chapter{Technické aspekty a architektúra}\label{ch:technicke-aspekty}

Pochopenie technickej architektúry Jiry je kľúčové pre efektívne nasadenie a využívanie nástroja. Táto kapitola sa venuje deployment modelom, základným konceptom dátového modelu a integračným možnostiam, ktoré spoločne tvoria základ pre praktické využitie opísané v~nasledujúcich kapitolách.

\section{Deployment modely}

Jira je dostupná v dvoch hlavných deployment modeloch, pričom každý prináša odlišné výhody a kompromisy. Výber medzi nimi závisí od veľkosti organizácie, bezpečnostných požiadaviek a dostupných zdrojov na správu infraštruktúry.

\subsection{Jira Cloud}

Cloud verzia predstavuje SaaS (Software as a Service) model, kde celú infraštruktúru spravuje spoločnosť Atlassian. Kľúčovou výhodou sú automatické aktualizácie~-- nové funkcie a~bezpečnostné záplaty sa nasadzujú kontinuálne bez manuálneho zásahu. Infraštruktúra sa automaticky škáluje podľa záťaže a~dátové centrá po celom svete zabezpečujú nízku latenciu. Z~ekonomického hľadiska Cloud verzia znižuje celkové náklady na vlastníctvo (TCO), keďže eliminuje potrebu správy vlastnej infraštruktúry. Platforma disponuje compliance certifikáciami vrátane SOC~2 Type~II, ISO~27001 a~je pripravená na plnenie požiadaviek GDPR \cite{atlassian_trust2024}.

Cloud verzia využíva multi-tenant architektúru, kde sú dáta rôznych organizácií logicky izolované, ale zdieľajú spoločnú infraštruktúru. To prináša ekonomické výhody, ale môže vyvolávať obavy u organizácií so špecifickými bezpečnostnými požiadavkami.

\subsection{Jira Data Center}

Data Center je self-hosted riešenie určené pre veľké organizácie s~vysokými nárokmi na dostupnosť a~kontrolu. Cluster architektúra s~aktívnymi uzlami zabezpečuje vysokú dostupnosť (HA) s~minimálnym prestojom. Organizácia si ponecháva plnú kontrolu nad infraštruktúrou, aktualizáciami i~zabezpečením, pričom dáta zostávajú v~on-premise prostredí. Tento model zároveň umožňuje hlbšie úpravy a~integráciu s~internými systémami.

Architektúru oboch deployment modelov porovnáva Obr.~\ref{fig:deployment}, z~ktorého sú zrejmé rozdiely v~správe infraštruktúry a~dátovom toku.

\begin{figure}[ht]
\centering
% TODO: vložiť obrázok -- diagram porovnávajúci Cloud (multi-tenant, Atlassian spravuje)
% a Data Center (cluster, zákazník spravuje) architektúru. Zobraziť load balancer,
% aplikačné uzly, databázu, zdieľané úložisko. Zdroj: vlastná schéma.
\fbox{\parbox{0.8\textwidth}{\centering\vspace{2cm}[TODO: vložiť obrázok]\vspace{2cm}}}
\caption{Porovnanie architektúry Cloud a~Data Center deploymentu}
\label{fig:deployment}
\end{figure}

\begin{table}[ht]
\centering
\caption{Porovnanie deployment modelov Jira}
\label{tab:deployment-comparison}
\begin{tabular}{|l|c|c|}
\hline
\textbf{Kritérium} & \textbf{Cloud} & \textbf{Data Center} \\
\hline
Správa infraštruktúry & Atlassian & Zákazník \\
Aktualizácie & Automatické & Manuálne \\
Škálovanie & Automatické & Manuálne \\
Počiatočné náklady & Nízke & Vysoké \\
Kontrola nad dátami & Limitovaná & Plná \\
High Availability & Zahrnuté & Vyžaduje konfiguráciu \\
Compliance & Štandardné & Customizovateľné \\
\hline
\end{tabular}
\end{table}

Z~porovnania v~Tabuľke~\ref{tab:deployment-comparison} vyplýva, že Cloud verzia je vhodnejšia pre organizácie, ktoré uprednostňujú nízke počiatočné náklady a~minimálnu správu infraštruktúry. Na druhej strane, Data Center je určený pre organizácie s~prísnymi regulačnými požiadavkami na umiestnenie dát alebo s~potrebou hlbokej customizácie, pre ktoré je plná kontrola nad prostredím kľúčová. V~praxi väčšina nových zákazníkov volí Cloud, zatiaľ čo Data Center zostáva relevantný predovšetkým vo finančnom sektore, zdravotníctve a~vládnych organizáciách.

\section{Základné koncepty a terminológia}

Dátový model Jiry je postavený na hierarchii objektov, ktoré umožňujú štruktúrované riadenie práce. Pochopenie týchto konceptov je nevyhnutné pre efektívne využívanie nástroja v~každodennej praxi.

\subsection{Issue (Úloha)}

Issue (úloha) je základná jednotka práce v~Jire. Každé issue je identifikované unikátnym kľúčom vo formáte PROJEKT-ČÍSLO (napr. PROJ-123) a~obsahuje krátky súhrn, detailný popis využívajúci rich text formátovanie a~zaradenie do konkrétneho typu (Bug, Story, Task, Epic, Sub-task). Ďalšími atribútmi sú aktuálny stav vo workflow (pracovnom toku), teda napríklad To~Do, In~Progress alebo Done, priorita relatívna k~iným issues, assignee (zodpovedná osoba) a~reporter (osoba, ktorá issue vytvorila). Nad rámec týchto štandardných polí Jira umožňuje definovať vlastné atribúty (custom fields) pre špecifické potreby organizácie.

\subsection{Projekt}

Projekt v~Jire združuje súvisiace issues a~definuje pre ne spoločný rámec: použité typy issues, workflow pre jednotlivé typy, prístupové práva a~oprávnenia, schémy notifikácií a~používané agilné dosky.

\subsection{Workflow}

Workflow (pracovný tok) definuje životný cyklus issue~-- povolené stavy a~prechody medzi nimi. Jira umožňuje vytvárať komplexné workflows s~podmienkami (conditions), ktoré určujú, kto môže vykonať prechod; validátormi (validators), ktoré kontrolujú splnenie predpokladov; a~post funkciami (post-functions), ktoré automaticky vykonajú akcie po prechode.

Príklad takéhoto workflow pre sledovanie chýb (bugov) zachytáva Obr.~\ref{fig:workflow}. Na diagrame je vidieť typické stavy ako Open, In~Progress, In~Review a~Closed spolu s~podmienkami prechodov medzi nimi.

\begin{figure}[ht]
\centering
% TODO: vložiť obrázok -- stavový diagram pre Bug workflow: stavy Open -> In Progress
% -> In Review -> Closed, s možnosťou Reopen z Closed do Open, s podmienkami
% a post-functions na prechodoch. Zdroj: vlastný screenshot z Jira workflow editora.
\fbox{\parbox{0.8\textwidth}{\centering\vspace{2cm}[TODO: vložiť obrázok]\vspace{2cm}}}
\caption{Príklad workflow pre Bug tracking s~jednotlivými stavmi a~prechodmi}
\label{fig:workflow}
\end{figure}

\subsection{Šprint a Verzia}

\textbf{Šprint} (iterácia) je časovo ohraničená iterácia v~Scrum metodike, typicky trvá 1--4 týždne. Jira umožňuje plánovanie issues do šprintov, sledovanie velocity (rýchlosti tímu) a~burndown chartov (grafov zostávajúcej práce) a~automatické presuny nedokončených issues do nasledujúceho šprintu.

\textbf{Verzia} (Release) reprezentuje release cyklus produktu. Umožňuje plánovanie funkcionalít pre konkrétne vydania, automatické generovanie release notes a~sledovanie progresu smerom k~releasu.

\section{Integrácie a ekosystém}

Sila Jiry spočíva nielen v samotnom nástroji, ale aj v rozsiahlom ekosystéme integrácií. Práve prepojiteľnosť s~externými nástrojmi robí z~Jiry centrálny uzol vývojového procesu, nie izolovanú aplikáciu.

\subsection{Atlassian platforma}

Jira sa natívne integruje s~ďalšími produktami Atlassian. S~Confluence umožňuje prepojenie issues s~dokumentáciou a~znalostnými bázami, s~Bitbucketom automatické linkovanie commitov, pull requestov a~buildov. Karty v~Trelle je možné synchronizovať s~Jira issues a~prostredníctvom Opsgenie sa Jira napája na incident management.

\subsection{REST API}

Jira poskytuje komplexné REST API umožňujúce vytváranie, čítanie, aktualizáciu a~mazanie issues, správu projektov, verzií a~šprintov, vyhľadávanie pomocou JQL (Jira Query Language) a~webhooky pre real-time notifikácie.

Príklad API volania pre vytvorenie issue \cite{atlassian_api2024}:

\begin{verbatim}
POST /rest/api/3/issue
{
  "fields": {
    "project": { "key": "PROJ" },
    "summary": "Nová úloha z API",
    "issuetype": { "name": "Task" }
  }
}
\end{verbatim}

\subsection{Atlassian Marketplace}

Atlassian Marketplace obsahuje tisíce rozšírení (apps) tretích strán~-- od pokročilých reportingových nástrojov (eazyBI, Structure), cez integrácie s~CI/CD nástrojmi (Jenkins, GitLab, Azure DevOps) a~rozšírenia pre testovanie (Zephyr, Xray), až po nástroje pre time tracking a~resource management. Ukážku rozhrania Marketplace zobrazuje Obr.~\ref{fig:marketplace}.

Podľa interných štatistík spoločnosti Atlassian, priemerná organizácia používa 8--12 marketplace aplikácií spolu s core Jira funkcionalitou \cite{atlassian_ecosystem2024}.

\begin{figure}[ht]
\centering
% TODO: vložiť obrázok -- screenshot Atlassian Marketplace: vyhľadávanie doplnkov,
% kategórie (reporting, CI/CD, testing, time tracking), hodnotenia, cena.
% Zdroj: vlastný screenshot z marketplace.atlassian.com.
\fbox{\parbox{0.8\textwidth}{\centering\vspace{2cm}[TODO: vložiť obrázok]\vspace{2cm}}}
\caption{Rozhranie Atlassian Marketplace s~kategóriami rozšírení}
\label{fig:marketplace}
\end{figure}
