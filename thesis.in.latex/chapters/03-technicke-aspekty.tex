% !TEX root = ../thesis.tex

\chapter{Technické aspekty a architektúra}\label{ch:technicke-aspekty}

Pochopenie technickej architektúry Jiry je kľúčové pre efektívne nasadenie a~využívanie nástroja. Táto kapitola sa venuje deployment modelom, základným konceptom dátového modelu a~integračným možnostiam.

\section{Deployment modely}

Jira je dostupná v~dvoch hlavných deployment modeloch, pričom každý prináša odlišné výhody a~kompromisy.

\subsection{Jira Cloud}

Cloud verzia predstavuje SaaS (Software as a~Service) model, kde je celá infraštruktúra spravovaná spoločnosťou Atlassian. Hlavné charakteristiky zahŕňajú:

\begin{itemize}
    \item \textbf{Automatické aktualizácie} -- Nové funkcie a~bezpečnostné záplaty sú nasadzované kontinuálne bez potreby manuálneho zásahu
    \item \textbf{Škálovateľnosť} -- Infraštruktúra sa automaticky prispôsobuje záťaži
    \item \textbf{Globálna dostupnosť} -- Dátové centrá po celom svete zabezpečujú nízku latenciu
    \item \textbf{Nižšie TCO} -- Eliminovaná potreba správy vlastnej infraštruktúry
    \item \textbf{Compliance certifikácie} -- SOC 2 Type II, ISO 27001, GDPR ready~\cite{atlassian_trust2024}
\end{itemize}

Cloud verzia využíva multi-tenant architektúru, kde sú dáta rôznych organizácií logicky izolované, ale zdieľajú spoločnú infraštruktúru. To prináša ekonomické výhody, ale môže vyvolávať obavy u~organizácií so špecifickými bezpečnostnými požiadavkami.

\subsection{Jira Data Center}

Data Center je self-hosted riešenie určené pre veľké organizácie s~vysokými nárokmi na dostupnosť a~kontrolu:

\begin{itemize}
    \item \textbf{Vysoká dostupnosť (HA)} -- Cluster architektúra s~aktívnymi uzlami zabezpečuje minimálny downtime
    \item \textbf{Plná kontrola} -- Organizácia má kontrolu nad infraštruktúrou, aktualizáciami a~zabezpečením
    \item \textbf{On-premise deployment} -- Dáta zostávajú v~infraštruktúre organizácie
    \item \textbf{Customizácia} -- Možnosť hlbších úprav a~integrácie s~internými systémami
\end{itemize}

[Obr. 3: Porovnanie architektúry Cloud a~Data Center deploymentu]

\begin{table}[ht]
\centering
\caption{Porovnanie deployment modelov Jira}
\label{tab:deployment-comparison}
\begin{tabular}{|l|c|c|}
\hline
\textbf{Kritérium} & \textbf{Cloud} & \textbf{Data Center} \\
\hline
Správa infraštruktúry & Atlassian & Zákazník \\
Aktualizácie & Automatické & Manuálne \\
Škálovanie & Automatické & Manuálne \\
Počiatočné náklady & Nízke & Vysoké \\
Kontrola nad dátami & Limitovaná & Plná \\
High Availability & Zahrnuté & Vyžaduje konfiguráciu \\
Compliance & Štandardné & Customizovateľné \\
\hline
\end{tabular}
\end{table}

\section{Základné koncepty a terminológia}

Dátový model Jiry je postavený na hierarchii objektov, ktoré umožňujú štruktúrované riadenie práce.

\subsection{Issue (Úloha)}

Issue je základná jednotka práce v~Jire. Každé issue má:
\begin{itemize}
    \item \textbf{Kľúč} -- Unikátny identifikátor vo formáte PROJEKT-ČÍSLO (napr. PROJ-123)
    \item \textbf{Súhrn} -- Krátky názov popisujúci úlohu
    \item \textbf{Popis} -- Detailná špecifikácia využívajúca rich text formátovanie
    \item \textbf{Typ} -- Kategorizácia (Bug, Story, Task, Epic, Sub-task...)
    \item \textbf{Stav} -- Aktuálna pozícia vo workflow (To Do, In Progress, Done...)
    \item \textbf{Priorita} -- Dôležitosť relatívne k~iným issues
    \item \textbf{Assignee} -- Osoba zodpovedná za vyriešenie
    \item \textbf{Reporter} -- Osoba, ktorá issue vytvorila
    \item \textbf{Custom fields} -- Používateľom definované atribúty
\end{itemize}

\subsection{Projekt}

Projekt združuje súvisiace issues a~definuje:
\begin{itemize}
    \item použité typy issues,
    \item workflow pre jednotlivé typy,
    \item prístupové práva a~oprávnenia,
    \item schémy notifikácií,
    \item používané agilné dosky.
\end{itemize}

\subsection{Workflow}

Workflow definuje životný cyklus issue -- povolené stavy a~prechody medzi nimi. Jira umožňuje vytvárať komplexné workflows s:
\begin{itemize}
    \item podmienkami (conditions) -- kto môže vykonať prechod,
    \item validátormi (validators) -- kontroly pred prechodom,
    \item post funkcami (post-functions) -- automatické akcie po prechode.
\end{itemize}

[Obr. 4: Príklad workflow pre Bug tracking s~jednotlivými stavmi a~prechodmi]

\subsection{Šprint a Verzia}

\textbf{Šprint} je časovo ohraničená iterácia v~Scrum metodike, typicky 1--4 týždne. Jira umožňuje:
\begin{itemize}
    \item plánovanie issues do šprintov,
    \item sledovanie velocity a~burndown,
    \item automatické presuny nedokončených issues.
\end{itemize}

\textbf{Verzia} (Release) reprezentuje release cyklus produktu a~umožňuje:
\begin{itemize}
    \item plánovanie funkcionalít pre konkrétne vydania,
    \item generovanie release notes,
    \item sledovanie progresu k~releasu.
\end{itemize}

\section{Integrácie a ekosystém}

Sila Jiry spočíva nielen v~samotnom nástroji, ale aj v~rozsiahlom ekosystéme integrácií.

\subsection{Atlassian platforma}

Jira sa natívne integruje s~ďalšími produktami Atlassian:
\begin{itemize}
    \item \textbf{Confluence} -- Prepojenie issues s~dokumentáciou a~znalostnými bázami
    \item \textbf{Bitbucket} -- Automatické linkovaniecommitov, pull requestov a~buildov
    \item \textbf{Trello} -- Synchronizácia kariet s~Jira issues
    \item \textbf{Opsgenie} -- Napojenie na incident management
\end{itemize}

\subsection{REST API}

Jira poskytuje komplexné REST API umožňujúce:
\begin{itemize}
    \item vytváranie, čítanie, aktualizáciu a~mazanie issues,
    \item správu projektov, verzií a~šprintov,
    \item vyhľadávanie pomocou JQL (Jira Query Language),
    \item webhooky pre real-time notifikácie.
\end{itemize}

Príklad API volania pre vytvorenie issue~\cite{atlassian_api2024}:

\begin{verbatim}
POST /rest/api/3/issue
{
  "fields": {
    "project": { "key": "PROJ" },
    "summary": "Nová úloha z API",
    "issuetype": { "name": "Task" }
  }
}
\end{verbatim}

\subsection{Atlassian Marketplace}

Atlassian Marketplace obsahuje tisíce rozšírení (apps) tretích strán, vrátane:
\begin{itemize}
    \item pokročilých reportingových nástrojov (eazyBI, Structure),
    \item integrácie s~CI/CD nástrojmi (Jenkins, GitLab, Azure DevOps),
    \item rozšírení pre testovanie (Zephyr, Xray),
    \item nástrojov pre time tracking a~resource management.
\end{itemize}

Podľa štatistík Atlassian, priemerná organizácia používa 8--12 marketplace aplikácií spolu s~core Jira funkcionalitou~\cite{atlassian_ecosystem2024}.
