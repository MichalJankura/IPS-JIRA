% !TEX root = ../thesis.tex

\chapter{Využitie nástroja Jira v~projektovom cykle}\label{ch:vyuzitie}

Táto kapitola detailne analyzuje praktické využitie Jiry v~jednotlivých fázach a~aktivitách agilného projektového riadenia. Zameriame sa na konkrétne funkcionality a~pracovné postupy, ktoré efektívne podporujú každodennú prácu tímov.

\section{Plánovanie a backlog management}

Efektívne plánovanie je základom úspešného projektu. Jira poskytuje robustné nástroje pre správu produktového backlogu a~plánovanie iterácií.

\subsection{Produktový backlog}

Produktový backlog v~Jire reprezentuje prioritizovaný zoznam všetkej práce, ktorá má byť na projekte vykonaná. Klúčové aspekty práce s~backlogom zahŕňajú:

\textbf{Hierarchická štruktúra} -- Jira umožňuje organizovať prácu do hierarchie:
\begin{enumerate}
    \item \textbf{Iniciatívy} -- Strategické ciele na najvyššej úrovni
    \item \textbf{Epiky} -- Veľké funkcionality rozkladané na menšie časti
    \item \textbf{User Stories/Tasky} -- Konkrétne pracovné položky
    \item \textbf{Sub-tasky} -- Detailné kroky pre implementáciu
\end{enumerate}

\textbf{Prioritizácia} -- Backlog view umožňuje:
\begin{itemize}
    \item drag-and-drop preusporiadanie podľa priority,
    \item filtrovanie pomocou JQL dotazov,
    \item bulk operácie pre hromadné zmeny,
    \item vizualizáciu dependencies medzi items.
\end{itemize}

\textbf{Refinement} -- Jira podporuje backlog grooming sessiony prostredníctvom:
\begin{itemize}
    \item story point odhady s~planning poker integráciou,
    \item priradzovania epikov a~verzií,
    \item pridávania acceptance criteria do opisu,
    \item flagovania items vyžadujúcich pozornosť.
\end{itemize}

[Obr. 5: Ukážka backlog view s~hierarchickou štruktúrou a~odhadmi]

\subsection{Sprint planning}

Sprint planning v~Jire prebieha nasledovne:

\begin{enumerate}
    \item \textbf{Vytvorenie šprintu} -- Definovanie názvu, dátumov začiatku a~konca, šprintového cieľa
    \item \textbf{Naplnenie šprintu} -- Presun items z~backlogu do šprintu pomocou drag-and-drop
    \item \textbf{Kapacitné plánovanie} -- Sledovanie celkových story points vs. velocity tímu
    \item \textbf{Štart šprintu} -- Aktivácia šprintu a~začiatok práce
\end{enumerate}

Jira poskytuje metriky pre informované rozhodovanie:
\begin{itemize}
    \item \textbf{Velocity chart} -- Historická kapacita tímu naprieč šprintami
    \item \textbf{Commitment vs. Completion} -- Pomer plánovanej a~dokončenej práce
    \item \textbf{Issue count} -- Počet items pre manažment WIP (Work In Progress)
\end{itemize}

\subsection{Roadmap a dlhodobé plánovanie}

Pre strategické plánovanie Jira ponúka:

\textbf{Basic Roadmap} (zahrnutý v~Jira Software) umožňuje:
\begin{itemize}
    \item vizualizáciu epikov na časovej osi,
    \item sledovanie dependencies medzi epikmi,
    \item zdieľanie roadmapy so stakeholdermi.
\end{itemize}

\textbf{Advanced Roadmap} (Jira Premium) pridáva:
\begin{itemize}
    \item plánovanie naprieč viacerými tímami a~projektami,
    \item what-if scenáre a~kapacitné modelovanie,
    \item automatické prepisovanie plánov pri zmenách,
    \item sledovanie rizík a~blockerov.
\end{itemize}

\section{Správa úloh a issue tracking}

Každodenná práca s~issues je jadrom používania Jiry. Efektívna správa úloh vyžaduje pochopenie best practices a~dostupných nástrojov.

\subsection{Životný cyklus issue}

Typický životný cyklus issue prechádza nasledujúcimi fázami:

\begin{enumerate}
    \item \textbf{Vytvorenie} -- Reporter vytvorí issue s~potrebnými detailmi. Dobre napísané issue obsahuje:
    \begin{itemize}
        \item jasný, akčný summary,
        \item detailný popis s~kontextom a~acceptance criteria,
        \item správny typ a~prioritu,
        \item prípadné prílohy (snímky obrazovky, logy).
    \end{itemize}
    
    \item \textbf{Triáž} -- Product Owner alebo tím leader:
    \begin{itemize}
        \item validuje relevantnosť issue,
        \item nastavuje prioritu,
        \item priraďuje do epiku a~verzie,
        \item odhaduje komplexitu (story points).
    \end{itemize}
    
    \item \textbf{Plánovanie} -- Issue je zaradené do šprintu na základe:
    \begin{itemize}
        \item priority a~business value,
        \item dostupnej kapacity tímu,
        \item technických dependencies.
    \end{itemize}
    
    \item \textbf{Implementácia} -- Vývojár pracuje na issue:
    \begin{itemize}
        \item prechod do stavu In Progress,
        \item prepojenie s~git branchom a~commitmi,
        \item dokumentácia technických rozhodnutí v~komentároch,
        \item logovanie odpracovaného času.
    \end{itemize}
    
    \item \textbf{Review} -- Kontrola kvality:
    \begin{itemize}
        \item code review cez prepojený pull request,
        \item testovanie podľa acceptance criteria,
        \item QA validácia pre produkčné releasy.
    \end{itemize}
    
    \item \textbf{Dokončenie} -- Prechod do stavu Done:
    \begin{itemize}
        \item splnenie definition of done,
        \item automatické notifikácie stakeholderom,
        \item aktualizácia release notes.
    \end{itemize}
\end{enumerate}

\subsection{JQL -- Jira Query Language}

JQL je výkonný dotazovací jazyk pre vyhľadávanie a~filtrovanie issues. Syntax kombinuje polia, operátory a~hodnoty~\cite{atlassian_jql2024}:

\begin{verbatim}
project = "PROJ" AND status = "In Progress" 
AND assignee = currentUser() 
ORDER BY priority DESC
\end{verbatim}

Pokročilé JQL dotazy umožňujú:
\begin{itemize}
    \item vyhľadávanie podľa vlastných polí,
    \item použitie funkcií (startOfWeek(), membersOf()),
    \item kombináciu podmienok s~AND, OR, NOT,
    \item ukladanie ako filtre pre opakované použitie.
\end{itemize}

Príklady užitočných JQL dotazov:
\begin{itemize}
    \item Všetky bugs s~vysokou prioritou: \texttt{type = Bug AND priority >= High}
    \item Úlohy bez assignee v~aktuálnom šprinte: \texttt{sprint in openSprints() AND assignee is EMPTY}
    \item Nedávno aktualizované: \texttt{updated >= -7d ORDER BY updated DESC}
\end{itemize}

\section{Agilné dosky -- Scrum a Kanban}

Vizuálne agilné dosky sú ústredným prvkom Jiry pre koordináciu tímovej práce~\cite{anderson2010}.

\subsection{Scrum Board}

Scrum board je optimalizovaný pre tímy pracujúce v~iteráciách:

\textbf{Štruktúra:}
\begin{itemize}
    \item Stĺpce reprezentujú workflow stavy (To Do, In Progress, In Review, Done)
    \item Karty zobrazujú issues s~kľúčovými informáciami (kľúč, summary, assignee, story points)
    \item Swimlanes môžu rozdeliť board podľa epikov, assignees alebo priorit
\end{itemize}

\textbf{Sprint-specific features:}
\begin{itemize}
    \item Sprint goal display pre zachovanie fokusuna cieľ
    \item Scope change indikátory (pridané/odobrané items po štarte)
    \item Days remaining countdown
    \item Quick filters pre filtrovanie kariet
\end{itemize}

[Obr. 6: Príklad Scrum boardu s~aktívnym šprintom]

\subsection{Kanban Board}

Kanban board je určený pre kontinuálny flow práce bez fixných iterácií:

\textbf{Charakteristiky:}
\begin{itemize}
    \item Žiadne šprint boundaries -- práca plynie kontinuálne
    \item WIP limity na stĺpcoch zabraňujú preťaženiu
    \item Column constraints vynucujú procesné pravidlá
    \item Cumulative flow diagram pre analýzu flow
\end{itemize}

\textbf{WIP limity} sú kľúčovou funkciou Kanban prístupu:
\begin{itemize}
    \item Nastavenie maximálneho počtu issues v~stĺpci
    \item Vizuálna indikácia pri prekročení limitu
    \item Podporuje pull-based systém práce
    \item Znižuje context switching a~zvyšuje throughput
\end{itemize}

\subsection{Konfigurácia dosiek}

Oba typy dosiek umožňujú rozsiahlu konfiguráciu:

\begin{itemize}
    \item \textbf{Columns} -- Mapovanie workflow stavov na stĺpce
    \item \textbf{Swimlanes} -- Horizontálne rozdelenie (queries, assignees, stories)
    \item \textbf{Card layout} -- Výber zobrazovaných polí na kartách
    \item \textbf{Quick filters} -- Predefinované JQL filtre pre rýchle prepínanie pohľadov
    \item \textbf{Card colors} -- Farebné kódovanie podľa priority, typu alebo vlastného kritéria
\end{itemize}

\section{Monitorovanie a reporting}

Dátami riadené rozhodovanie vyžaduje kvalitné reportingové nástroje. Jira poskytuje bohatú paletu reportov a~dashboardov.

\subsection{Štandardné reporty}

Jira obsahuje množstvo zabudovaných reportov:

\textbf{Scrum reporty:}
\begin{itemize}
    \item \textbf{Burndown Chart} -- Zobrazuje zostávajúcu prácu v~šprinte voči ideálnej línii
    \item \textbf{Velocity Chart} -- Historická kapacita tímu naprieč šprintami
    \item \textbf{Sprint Report} -- Súhrn šprintu vrátane scope changes
    \item \textbf{Burnup Chart} -- Kumulatívne dokončená práca
\end{itemize}

\textbf{Kanban reporty:}
\begin{itemize}
    \item \textbf{Cumulative Flow Diagram} -- Distribúcia issues naprieč stavmi v~čase
    \item \textbf{Control Chart} -- Cycle time a~lead time analýza
\end{itemize}

\textbf{Univerzálne reporty:}
\begin{itemize}
    \item \textbf{Created vs. Resolved} -- Trend vytvárania a~riešenia issues
    \item \textbf{Resolution Time} -- Štatistiky času potrebného na vyriešenie
    \item \textbf{Workload Pie Chart} -- Distribúcia práce medzi členov tímu
\end{itemize}

[Obr. 7: Príklad Burndown chartu s~ideálnou a~aktuálnou líniou]

\subsection{Dashboardy}

Dashboardy umožňujú agregáciu informácií z~rôznych zdrojov na jednom mieste:

\begin{itemize}
    \item \textbf{Gadgety} -- Modulárne komponenty zobrazujúce rôzne metriky
    \item \textbf{Filters} -- Zdieľané JQL dotazy ako základ gadgetov
    \item \textbf{Sharing} -- Dashboardy môžu byť privátne, zdieľané alebo globálne
    \item \textbf{Wall mode} -- Full-screen zobrazenie pre team rooms
\end{itemize}

Príklady užitočných gadgetov:
\begin{itemize}
    \item Filter Results -- Zoznam issues podľa filtra
    \item Pie Chart -- Distribúcia issues podľa vybraného poľa
    \item Activity Stream -- Posledné aktivity na projekte
    \item Sprint Burndown -- Aktuálny stav šprintu
\end{itemize}

\subsection{Advanced Analytics}

Pre hlbšiu analýzu poskytuje Jira Cloud:

\textbf{Insights} -- Zabudované analytické nástroje:
\begin{itemize}
    \item Deployment frequency tracking
    \item Lead time for changes
    \item Change failure rate
    \item Mean time to recovery (DORA metriky)~\cite{forsgren2018}
\end{itemize}

\textbf{Marketplace riešenia} pre pokročilú analytiku:
\begin{itemize}
    \item eazyBI -- BI nástroj s~OLAP kockami
    \item Structure -- Hierarchická vizualizácia
    \item BigPicture -- Portfolio management a~gantt charty
\end{itemize}
